
\documentclass[12pt]{article}
\title{Asymptotic Behavior in Rational Functions}
\author{Will Hanstedt}
\date{May 22, 2018}
\usepackage{amssymb}
\usepackage[utf8]{inputenc}

\begin{document}

\maketitle

\begin{abstract}
The behavior of rational functions around asymptotes, especially 		% Scan all for consistent use of terminology
horizontal and so-called ``oblique" asymptotes, is poorly taught 
and poorly understood in the context of high school algebra. It is
necessary to define a new approach to such topics, for the purpose
of better understanding and simpler computation on the part of 
students. In this, we will outline an alternative method for 
finding asymptotes for students and a new way to think about 
rational expressions for teachers.
\end{abstract}

\section{Background}
\subsection{Rational Functions}

A rational function is any function that can be represented as a 
fraction where the numerator and denominator are 
polynomials.\footnote{Wikipedia}
This includes, but is not limited to, examples such as 

	\[\frac{1}{x^{3}}\ \ \ \ 
\textnormal{or}\ \ \ \  
	\frac{x-2}{x+2}\ \ \ \ 
\textnormal{or}\ \ \ \ 
	\frac{4x^{4}+3x^{3}+2x^{2}+x}{x-1}\]

Despite the fact that polynomials themselves may have domains 
that encompass the set of real numbers $\mathbb{R}$ (and indeed the set of
complex numbers $\mathbb{C}$ as well), many rational functions have restrictions 
in their domains. This can be demonstrated by inputting $-2$ into
	\(\frac{x-2}{x+2}\).
The result is undefined, and the surrounding function behaves asymptotically.\\

The tendency of the funtion towards infinity or negative infinity around
this $x$ value is refered to as a vertical asymptote, because the function,
as one moves up or down the output values, approaches the asymptotic input,
but does not reach it. This accelerated growth of the input occurs because,
as the denominator of the fraction approaches 0, the fraction itself becomes
larger and larger.\\

Likewise, a graph of the same function reveals that, as we follow the graph
towards \(x = +\infty \), the output tends towards \(f(x) = 1\). Logically, this
is sensible; as the $x$ in the numerator and the $x$ in the denominator both
increase, the effects of the $\pm 2$ become negligible. Note, however, that, 
just as with the vertical asymptote, the output of 1 is never reached, because 
of the slight difference between the numerator and denominator. \\

\subsection{Motivation}

The method commonly taught for finding asymptotes is patchwork at
best. Finding vertical asymptotes is accomplished through a method 
similar to the one described here, so it will be covered later. But to 
discover horizontal or oblique asymptotes, the process is more 
complicated.\\

First, students note the highest powers of the input variable in both
the numerator and denominator of the function. If the power of the 
numerator is (exactly) one power higher, then it is a slant asymptote. If one wants to 
find the equation of that asymptote, one then must do long division
(as described below). If the powers of the numerator and 
denominator are the same or the numerator has the lesser power,
it is a horizontal asymptote. In the case of the former, the ratio of
the coefficients gives the output value of the asymptote. In the case 
of the latter, the asymptote is at $f(x) = 0$.\\

Unfortunately, this method for determining the presence of 
non-vertical asymptotes relies entirely on the memory of the student. 
They must remember two different rules and when to use each of
three methods to find the asymptote. 
It also depends on the end     % Effect of long division on holes?
behavior of the graphs, which in turn relies on computation involving infinities, a 
concept with unnecesary complications for the high school level. \\			%Use in graphing

These shortcomings of the current method couple with a few benefits
of a more reliable method. For one, a better way of finding asymptotes
will assist in graphing (the bane of any algebra student's existence). 
In addition, this method can be better used to predict actual behavior 
around non-vertical asymptotes, as we will see later.\\

\subsection{Holes}			% Write this. Explain terminology and note important things

\section{The Two-Form Method}

In this section, we will introduce a new method for approaching 		%Differentiate btw "original" and "given"
rational functions in order to discover asymptotic behavior. As the
name implies, this method relies on converting the given form of
a rational funtion into two different, distinct forms, in order to 
obtain information about asymptotic behavior.\\

\subsection{Factored Form}

We start with the form closest to the conventional method, the 
factored version of the rational function. This will give us domain 
restrictions and zeroes of the function. Obtaining this form is 
relatively simply: we just find the factored forms of the numerator
and denominator of the function. For instance, 
	\[\frac{x^2-3x+2}{x^2-4x+3} \ \ \ \ 
	\textnormal{becomes} \ \ \ \ 
	\frac{(x-2)(x-1)}{(x-3)(x-1)}\]

Note that the original form is functionally identical to the factored
form. Graphs of the two will demonstrate this. Importantly, this
means that, as any one of the individual factors approaches
zero, the numerator or denominator will also approach
zero. As the numerator of the function approaches zero, the function
itself will approach 
	\(\displaystyle\frac{0}{g(x)}\)
, meaning that, unless whatever $g(x)$ is evaluates to zero at that same input 		% Space before comma and format of fraction?
coordinate, the function will have a zero at that input. \\

As the denominator of the function approaches zero, however, the 
function will approach 
$\pm\infty$.\footnote{Except, as we will see, in the case of ``holes."} 
Thus, each zero of the polynomial in the denominator is a vertical asymptote.\\

\subsection{Divided Form}

This second form for the function is the key part of the new 
approach. Essentially, we will  express the rational function
as the sum of a polynomial and a remainder, an 
indivisible rational function. This will, most importantly, always 
identify horizontal or oblique asymptotes, without the need for
the many conditions of the conventional method.\\

In order to obtain this form from the given, we simply use 		% Explain long division
polynomial long division. This leaves us with a polynomial
(the highest power of which will be the difference of the 
highest power of the numerator and the highest power of
the denominator of the given), and a remainder, a rational 
function that cannot be further divided (that is, the highest
power of its numerator is less than the highest power of 
its denominator). We may, for the time being, ignore the 
remainder and focus on the polynomial.\\

The polynomial part of the divided form is the non-vertical
asymptote. If one were to isolate that portion of the function		%image example
and graph it alongside the given, that would be clear. The power
of the input variable in the polynomial function determines the 
type of asymptote created, just as it would an actual polynomial. 
If it is raised to the first power, the
asymptote is oblique, while having no input variables present
creates a horizontal asymptote. As should be clear, this eliminates
the need for analyzing the given to discover the type of asymptote,
as most algebra students can better handle polynomials than 
rational functions themselves. \\

To better understand this, let us consider an example. Using		%formatting?
long division,
\begin{equation} 
	\frac{x^{3}+3x^{2}-10x-24}{x^{2}-3x+2}\ \ \ \ 
	\textnormal{reduces to}\ \ \ \ 
	 x+6 + \frac{6x-36}{x^{2}-3x+2}
\end{equation}

The asymptote can be described by $f(x) = x+6$. Because this	%indivisibles and y = 0
function has slope, the asymptote is oblique. Had we used the 
current method, we would have first had to check the powers
of the input variable, remembered the type of asymptote to 
which the positive difference corresponds, and then still do
long division to find the actual equation of the asymptote. The
method being proposed eliminates the need to remember the 
cases, because it relies on intuition of polynomials that students 
should already have.\\

One way to think about this method (for teachers or more invested		% Change word "transformation"
students) is as a transformation of a polynomial. The asymptote may 
be thought of as the ``original" function (note the distinction with the
``given" functions described earlier), and the irreducible rational 
expression that is added --- the remainder --- as the transformation 
upon the original function. After one adds the two (original and 		%talk about how powers work, how current method is constructed
transformation) together, the result may be transformed to a rational 
function in the form with which we are familiar (the ``given" form) by 
finding a common denominator between the two and multiplying the 	% How do holes appear?
original by an identity that will give it this denominator. \\

This transformation has a few consequences:
the asymptotes themselves. As the input approaches one of the domain 
restrictions in the transformational expression (provided it is not a ``hole"),
the tranformation part of the expression will grow, creating deviation from
the original in the transformed function. Thus, the rational function created
has the same vertical asymptotes as the transformation itself did. But,
as the input progresses further from the domain restrictions, the 
the transformation tends towards zero. This is because, since the 
tranformation is required to be irreducible by the method of long 
division used, the power of the denominator is smaller than that of
the numerator, and thus grows more rapidly than the numerator.
But, if this tendency towards zero is added to the original function,
it becomes a tendency towards the original function, or an asymptote.\\

It may be noted, if the example above were to be graphed, that	%Move to sec. 3?
the function itself actually crosses the oblique asymptote. This is 
explained by the distinction made between vertical and non-vertical
asymptotes: vertical asymptotes are created by domain restrictions
such as the ones described in 2.1, but horizontal and slant asymptotes
are created whenever a vertical asymptote is added, through the 
transformation described. Vertical asymptotes are the only type
whose implementation requires that the points on the asymptote
are never reached by the function. Because horizontal and oblique 
asymptotes are determined by the value of the remainder at a
given input, the asymptote can be crossed at the zeroes of the
transformation function. In the case of the above example, $x=6$
makes the remainder evaluate to zero, and thus the asymptote 
is crossed.\\

\section{Further Benefits}

This new method, and the accompanying interpretation, carry with
them some opportunities for generalization. For example, the 
old method only allows for a difference of powers (numerator
minus denominator) of one or less. This method, however, equips 
us to deal with other varieties of rational functions. Consider, for 		%figure out how formatting works
instance,
\begin{equation}
	\frac{x^{6}+2x^{5}-34x^{4}-52x^{3}+233x^{2}+50x-200}{x^{3}-5x^{2}-2x+24}
\end{equation}
(not, unfortunately, a very appealing equation). Factoring reveals it
to be equivalent to 
\begin{equation}
	\frac{(x+1)(x-1)(x-2)(x+4)(x-5)(x+5)}{(x+2)(x-3)(x-4)}
\end{equation}
while long division yields
\begin{equation}
	x^{3}+7x^{2}+3x-47+\frac{-164x^{2}-116x+928}{x^{3}-5x^{2}-2x+24}
\end{equation}
This is difficult enough to deal with (or graph) without the fact that 
the commonly used method for dealing with this simply cannot account 
for this type of problem. But looking at the two forms, we can clearly
see how to approach this problem. From our divided form, we find
that the asymptote has the equation \(x^{3}+7x^{2}+3x-47\). It is 
neither oblique nor horizontal, but a higher-powered --- a cubic, 
to be precise --- function. This is, for lack of a proper term, a 
curved asymptote. Furthermore, by solving the numerator of the 
remainder at zero, we find that the given function crosses the
asymptote at $x=\displaystyle\frac{-29\pm3\sqrt{4321}}{82}$.\footnote{Some
graphing utilities may display another crossing at $x=3$, but evaluating
this directly will reveal a vertical asymptote. }

\end{document}



